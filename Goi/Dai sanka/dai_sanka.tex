\documentclass[a4paper,10pt]{article}

\def\header#1#2#3#4#5#6{\pagestyle{empty}
\noindent
\begin{minipage}[t]{0.6\textwidth}
\begin{flushleft}
\bf #4\\% Fach eintragen
#5 %Namen
\end{flushleft}
\end{minipage}
\begin{minipage}[t]{0.4\textwidth}
\begin{flushright}
\bf #6\\
#2 %Datum eintragen
\end{flushright}
\end{minipage}

\begin{center}
{\Large\bf Vokabeln} %Blatt eintragen

%{(Abgabetermin #3)}
\end{center}
}

% Basic Packages
\usepackage[utf8]{inputenc}
\usepackage[top=2.2cm, right=2.2cm, bottom=2.2cm, left=2.2cm]{geometry}
\usepackage[german]{babel}
\usepackage{CJK}
\usepackage{longtable}
\usepackage{fontenc}
\usepackage{graphicx}

% Layout Packages
\usepackage{textcomp}
\usepackage{multicol}
\usepackage{ulem}
%\usepackage{wasysym}
\usepackage{pifont}
\pagenumbering{arabic}

\begin{document}
\header{Nr. 2 \\}{\today}{11.05.15}{JAPANISCH I}{}{SS 15}
\pagestyle{plain}
\begin{CJK}{UTF8}{min}
\begin{center}
\begin{longtable}{|p{4cm}|p{2cm}|p{8cm}|}
  \hline
  ごい & 語彙 & Vokabeln \\
  \hline \hline
  ソフィ	& & Sophie \\
  \hline
  ここ & & hier (Umgebung des Sprechers) \\
  \hline  
  そこ  & & da (Umgebung des Gespr\"achspartners) \\  
  \hline
  あそこ	& & dort dr\"uben (von beiden entfernt) \\
  \hline
  どこ & & wo \\
  \hline  
  こちら  & & hierhin, hier (h\"ofliche Entsprechung von ここ)\\
  \hline
  そちら	& & dahin, da (h\"ofliche Entsprechung von そこ)\\
  \hline
  あちら & & dorthin, dort dr\"uben (h\"ofliche Entsprechung von \newline あそこ)\\
  \hline
 どちら  & & wohin, wo (h\"ofliche Entsprechung von どこ)\\
  \hline
  きょうしつ & 教室 & Unterrichtsraum\\
  \hline
  しょくどう & 食堂  & Speisesaal, Kantine \\
  \hline
  じむしょ & 事務所 & B\"uro\\
  \hline  
  かいぎしつ & 会議室 & Konferenzraum\\
  \hline
  うけつけ &	受付 & Anmeldung, Rezeption\\
  \hline
  ロビー & & Lobby\\
  \hline  
  へや &	部屋 & Zimmer, Raum\\
  \hline
  トイレ (おてあらい) (お手洗い) & & Toilette, WC\\
  \hline
  かいだん &	階段 & Treppe\\
  \hline  
  エレベーター & & Aufzug\\
  \hline
  エスカレーター & & Rolltreppe\\
  \hline
  じどうはんばいき	& 自動販売機 & Verkaufsautomat\\
  \hline  
  でんわ & 電話  & Telefon, Anruf\\
  \hline
  [お]くに & [お]国	& [Ihr] Land \\
  \hline
  かいしゃ &	会社 & Firma\\
  \hline  
  うち & & Haus \\
  \hline
  くつ & 靴 & Schuhe \\
  \hline
  ネクタイ & & Krawatte\\
  \hline  
   ワイン & & Wein\\
  \hline
  うりば & 売り場 	& Verkaufsabteilung (in einem Kaufhaus)\\
  \hline
  ちか & 地下 & Untergeschoss\\
  \hline  
  ーかい (ーがい) & ─階 & -te Etage\\
  \hline
  なんがい & 何階	& welche Etage\\
  \hline
  ーえん	& ー円 & - Yen \\
  \hline  
   いくら & & wie viel (nur f\"ur Geld) \\
  \hline
  ひゃく	& 百 & Hundert \\
  \hline
  せん &	千 & Tausend \\
  \hline  
  まん & 万 & Zehntausend \\
  \hline
  すみません。&	& Entschuldigen Sie bitte!/Entschuldigung! (wenn man jmd. anspricht) \\
  \hline
  どうも。 & & Danke! \\
  \hline  
  いらつしゃいませ。&  & Willkommen!/Was kann ich f\"ur Sie tun? (Begr\"ußungsformel an Kunden, die ins Gesch\"aft gekommen sind)\\
  \hline
  [...を] 見せてください。&	& Zeigen Sie mir bitte [...]!\\
  \hline
  じゃ & & hm, nun, dann (wenn man die Worte des Gespr\"achspartners aufnimmt) \\
  \hline  
  [...を]ください &  & Geben Sie mir bitte [...]! (beim Einkaufen, Bestellen etc.)\\
  \hline
  イタリア &	& Italien\\
  \hline
  スイス & & Schweiz\\
  \hline  
  フランス & & Frankreich\\
  \hline
  ジャカルタ & & Jakarta\\
  \hline
  バンコク & & Bankok\\
  \hline  
  ベルリン & & Berlin\\
  \hline
\end{longtable}
\end{center}

\end{CJK}
\end{document}