\documentclass[a4paper,10pt]{article}

\def\header#1#2#3#4#5#6{\pagestyle{empty}
\noindent
\begin{minipage}[t]{0.6\textwidth}
\begin{flushleft}
\bf #4\\% Fach eintragen
#5 %Namen
\end{flushleft}
\end{minipage}
\begin{minipage}[t]{0.4\textwidth}
\begin{flushright}
\bf #6\\
#2 %Datum eintragen
\end{flushright}
\end{minipage}

\begin{center}
{\Large\bf Dai ikka: reibun} %Blatt eintragen

%{(Abgabetermin #3)}
\end{center}
}

% Basic Packages
\usepackage[utf8]{inputenc}
\usepackage[top=2.2cm, right=2.2cm, bottom=2.2cm, left=2.2cm]{geometry}
\usepackage[german]{babel}
\usepackage{CJK}
\usepackage[CJK, overlap]{ruby}
\usepackage{longtable}
\usepackage{fontenc}
\usepackage{graphicx}

% Layout Packages
\usepackage{textcomp}
\usepackage{multicol}
\usepackage{ulem}
%\usepackage{wasysym}
\usepackage{pifont}
\pagenumbering{arabic}

\begin{document}
\header{Nr. 2 \\}{\today}{11.05.15}{JAPANISCH I}{}{SS 15}
\pagestyle{plain}
\begin{CJK}{UTF8}{min}
\ruby{第}{だい}3\ruby{課}{か} 
\begin{center}
\begin{longtable}{|p{8cm}|p{8cm}|}
  \hline
  れいぶん & Beispiele\\
  \hline \hline
   \ruby{第}{だい}1\ruby{課}{か}& Lektion 1\\
  \hline \hline
  あなたは ソフィさん ですか。はい、わたしは ソフィ です。& Sind Sie Sophie? Ja, ich bin Sophie.\\
  \hline
  ソフィさんはかいしゃいんですか。いいえ、わたしは かいしゃいんじゃ ありません。がくせい です。& Sophie ist Angestellte. Nein, ich bin keine Angestellte. Ich bin Studentin.\\
 \hline
  
\end{longtable}
\end{center}

\end{CJK}
\end{document}